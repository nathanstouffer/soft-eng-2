\documentclass{article}

\usepackage[margin=1in]{geometry}
\usepackage{amsmath}
\usepackage{graphicx}
\usepackage{multicol}
\usepackage{fancyvrb}
\usepackage{bibentry}
\usepackage[shortlabels]{enumitem}
\usepackage{tikz}

\newcommand{\fig}[3]{ 
	\begin{figure}[h]
		\centering
		\caption{#3}
		\includegraphics[width=#2\textwidth]{pics/#1}
		\label{fig:#1}
	\end{figure} 
}

\newcommand{\nota}{\overline{a}}
\newcommand{\notb}{\overline{b}}
\newcommand{\notc}{\overline{c}}
\newcommand{\notd}{\overline{d}}
\newcommand{\fbar}{\overline{f}}

\begin{document}
\title{ESOF 422 - Homework 8}
\author{Nathan Stouffer}

\maketitle
\newpage
\section*{Question 1}

For Question 1, consider the following method.

\begin{verbatim}
public static int search (List list, Object element)
// Effects: if list or element is null, throw NullPointerException
//   else if element is in list, return an index
//     of element in the list;
//   else return -1
//   
//   for example, search ([3,3,1], 3) = either 0 or 1
//     search ([1,7,5], 2) = -1
\end{verbatim}
Answers will be based on the following characteristic partitioning. \\\\
Characteristic: Location of element in list \\
\indent \textbf{Block 1:} element is first entry in list \\
\indent \textbf{Block 2:} element is last entry in list \\
\indent \textbf{Block 3:} element is in some position other than first or last \\
\begin{enumerate}[(a)]
	\item ``Location of element in list'' fails the disjointness property when the input list is of length 1. For example, take the input to be search ([3], 3). The input is in both blocks 1 and 2. Additionally, the example given in the problem (search ([3,3,1], 3)) fails disjointness since it could be in blocks 1 and 3.
	\item ``Location of element in list'' fails the completeness property since it does not cover the case where the element is not in the list. Consider the example search ([1, 4, 5], 6). It does not reside in any block.s
	\item We now give new partitions that capture the intent of ``Location of element in list'' but do not suffer from completeness or disjointness problems. Although it should be noted that there are invalid combinations between characteristics.
\end{enumerate}
Characteristic: Length of list \\
\indent \textbf{Block 1:} list is null \\
\indent \textbf{Block 2:} list is length 0 \\
\indent \textbf{Block 3:} list is length 1 \\
\indent \textbf{Block 4:} list is length greater than 1 \\\\
Characteristic: Element in list \\
\indent \textbf{Block 1:} element is in list \\
\indent \textbf{Block 2:} element is not in list \\\\

\newpage
\section*{Question 2}

For Question 2, consider the following method.

\begin{verbatim}
public Set intersection (Set s1, Set s2)
// Effects: If s1 or s2 is null, throw NullPointerException
//   else return a (non null) Set equal to the intersection
//     of Sets s1 and s2
\end{verbatim}
Answers will be based one the following characteristic partitioning. \\\\
Characteristic: Validity of s1 \\
\indent \textbf{Block 1:} s1 = null \\
\indent \textbf{Block 2:} s1 = \{\} \\
\indent \textbf{Block 3:} s1 has at least on element \\\\
Characteristic: Relation between s1 and s2 \\
\indent \textbf{Block 1:} s1 and s2 represent the same set \\
\indent \textbf{Block 2:} s1 is a subset of s2 \\
\indent \textbf{Block 3:} s2 is a subset of s1 \\
\indent \textbf{Block 4:} s1 and s2 do not have any elements in common \\
\begin{enumerate}[(a)]
	\item The partition ``Validity of s1'' satisfies the completeness property. All possible inputs of s1 are coveerd.
	\item The partition ``Validity of s1'' satisfies the disjointness property; no two partitions overlap.
	\item The partition ``Relation between s1 and s2'' does not satisfy the completeness property. Consider s1 = \{ 0, 1 \} and s2 = \{ 1, 2 \}. They are not the same set, nor are they subsets of each other. Additionally, they do have elements in common, so this test set does not fit in any block.
	\item The partition ``Relation between s1 and s2'' does not satisfy the disjointness property. For example, take s1 = \{ 1, 3 \} and s2 = \{ 1, 3 \}. Then we have s1 and s2 representing the same set, s1 is a subset of s2, and s2 is a subset of s1. That is, this example fits into blocks 1, 2, and 3.
	\item We now give the number of test requirements ig Base Choice criterion were applied to the two partitions. In general, Base Choice requires $1 + \Sigma _{i=1} ^Q (B_i - 1)$ choices. In this scenario, we have $1 + 2 + 3 = 6$ test requirements.
	\item We are now asked to revise the characteristics to eliminate any problems that have been found. First consider the following definition of a proper subset: A is a subset of B if each element of A is an element of B and there exists an element of B not in A. Using this definition, we now give the following characteristics.
\end{enumerate}
Characteristic: Validity of s1 \\
\indent \textbf{Block 1:} s1 = null \\
\indent \textbf{Block 2:} s1 = \{\} \\
\indent \textbf{Block 3:} s1 has at least on element \\\\
Characteristic: Relation between s1 and s2 \\
\indent \textbf{Block 1:} s1 and s2 represent the same set \\
\indent \textbf{Block 2:} s1 is a proper subset of s2 \\
\indent \textbf{Block 3:} s2 is a proper subset of s1 \\
\indent \textbf{Block 4:} s1 and s2 (both nonempty) have elements in common but are not subsets of each other \\
\indent \textbf{Block 5:} s1 and s2 (both nonempty) do not have any elements in common 

\end{document}